\documentclass{article}

\usepackage[english]{babel}
\usepackage[utf8]{inputenc}
\usepackage[T1]{fontenc}

\usepackage{color}
\usepackage{enumerate}
\usepackage{amsmath}
\usepackage{hyperref}
\usepackage{multirow}
\usepackage{float}
\usepackage{graphicx}
\usepackage{mathtools}
\usepackage{amsfonts}
\usepackage[all]{xy}
\usepackage{xspace}
\usepackage{listings,multicol,multirow}
\usepackage{graphicx}
\usepackage{url}
\usepackage{amsmath}
\usepackage{amsfonts}
\usepackage{amsthm}
\usepackage{paralist}
\usepackage{enumitem}
\usepackage{pifont}
\usepackage{chngpage}
\usepackage{fancyhdr}
\usepackage{xcolor}
\usepackage[margin=1in]{geometry}

% Task, TODO and comment env
\newcommand{\comment}[1]{ {\color{blue} $\Rightarrow_{comment}$ #1} }
\newcommand{\taskinl}[1]{ {\color{red} $\bullet$ \textbf{TODO} #1} }
\newcommand{\task}{\item}
\newenvironment{TODO}{%
    \begin{center}
    \color{red}\textbf{Unfinished tasks:}
        \begin{enumerate}
}{%
        \end{enumerate}
    \end{center}
  }

\definecolor{OliveGreen}{rgb}{0,0.6,0}
\definecolor{Blue}{rgb}{0.0, 0.0, 0.6}
\definecolor{Red}{rgb}{0.6, 0.0, 0.0}

\newcommand{\EasyCrypt}{\textsf{EasyCrypt}\xspace}
\newcommand{\OCaml}{\textsf{OCaml}\xspace}
\newcommand{\FSTAR}{\textsf{F$\star$}\xspace}

\newcommand{\INDCPA}{\texttt{IND-CPA}\xspace}

\newtheorem{theorem}{Theorem}[section]
\newtheorem{lemma}[theorem]{Lemma}

% !TEX root = ./main.tex

%\newcommand{\result}{\mathsf{res}}
%\newcommand{\arrayVar}[2][]{{\mathbf{#2}_{#1}}}

\lstdefinelanguage{easycrypt}{
  style=easycrypt-default,
%  procnamekeys={op,pred,fun},
%  procnamestyle={\sffamily\itshape},
  keywordsprefix={'},
  morekeywords=[1]{unit,bool,int,real,list,matrix,option,distr,fset,array,true,false,res, Some, None},
  morekeywords=[2]{Pr,type,op,axiom,lemma,module,pred,const,declare},
  morekeywords=[3]{var,proc,let,in,return},
  morekeywords=[4]{while,if,then,else,fun},
  morekeywords=[5]{theory,end,clone,import,export,as,section,of,proving,with},
  morekeywords=[6]{forall,exists},
  morekeywords=[7]{idtac,change,beta,iota,zeta,logic,delta,simplify,congr,generalize,
                   pose,split,left,right,case,intros,cut,elim,apply,rewrite,elimT,subst,
                   progress,trivial,move, have, wp, rnd, call, skip, rcondt, rcondf},
  morekeywords=[8]{by,assumption,smt,reflexivity, done},
  morekeywords=[9]{first,last,do,try},
  morekeywords=[10]{int_bool_tuple, t, tree},
%  morekeywords=[11]{Rand_t,Distinguisher_t,Evaluator_t,Simulator_t,FRand_t,Adversary_t,MProver_t,MVerifier_t,RandP_t,RandV_t},
%  morekeywords=[12]{Idle,api_input,api_output,api_operator,API,global_state,AStep},
%  morekeywords=[13]{group,t,skey,pkey,plaintext,ciphertext,msg,cipher,Scheme,Adv},
%  moredirectives={prover,print}, % Incomplete
  morecomment=[n][\itshape\color{gray}]{(*}{*)},
  morecomment=[n][\bfseries\color{darkgray}]{(**}{*)}
}

\lstdefinestyle{easycrypt-default}{
  columns=fullflexible,
  captionpos=b,
  frame=tb,
  breaklines=true,
  %xleftmargin=.1\textwidth,
  %xrightmargin=.1\textwidth,
  rangebeginprefix={(**\ begin\ },
  rangeendprefix={(**\ end\ },
  rangesuffix={\ *)},
  includerangemarker=false,
  basicstyle=\small\ttfamily,
  identifierstyle={},
  keywordstyle=[1]{\itshape\color{OliveGreen}},
  keywordstyle=[2]{\bfseries\color{Blue}},
  keywordstyle=[3]{\bfseries\color{Red}},
  keywordstyle=[4]{\bfseries\color{Red}},
  keywordstyle=[5]{\bfseries\color{Blue}},
  keywordstyle=[6]{\itshape\color{Blue}},
  keywordstyle=[7]{\color{Blue}},
  keywordstyle=[8]{\color{Red}},
  keywordstyle=[9]{\color{OliveGreen}},
  keywordstyle=[10]{\itshape\color{OliveGreen}},
  keywordstyle=[11]{\itshape\color{OliveGreen}},
  keywordstyle=[12]{\itshape\color{OliveGreen}},
  keywordstyle=[13]{\itshape\color{OliveGreen}},
  keywordstyle=[14]{\itshape\color{OliveGreen}},
  literate={phi}{{$\!\phi\,$}}1
           {phi1}{{$\!\phi_1$}}1
           {phi2}{{$\!\phi_2$}}1
           {phi3}{{$\!\phi_3$}}1
           {phin}{{$\!\phi_n$}}1
}

\lstdefinestyle{easycrypt-pretty}{
    basicstyle=\small\ttfamily,
    literate={\\in}{{$\mathrel{\in}\;$}}1
              {:=}{{$\mathrel{\gets}$}}1
              {<=}{{$\mathrel{\leq} \; \;$}}1
              {<=}{{$\mathrel{\leq} \; \;$}}1
              {>=}{{$\mathrel{\geq}$}}1
              {<>}{{$\mathrel{\neq} \; \;$}}1
              {=\$}{{$\stackrel{\$}{\gets}\;$}}1
              {forall}{{$\forall\;$}}1
              {exists}{{$\exists\;$}}1
              {->}{{$\rightarrow\;$}}1
              {<r}{$\mathrel{< \! \! \$ \;}$}1
              {<p}{$< \! \! @\;$}1
              {<-}{{$\leftarrow\;$}}1
              {<->}{{$\leftrightarrow\;$}}1
              {<=>}{{$\Leftrightarrow\;$}}1
              {=>}{{$\Rightarrow\;$}}1
              {==>}{{$\Rrightarrow\;$}}1
              {\/\\}{{$\wedge\;$}}1
              {\\\/}{{$\vee\;$}}1
              {.\[}{{[}}1
              {''ora}{{$\mathrel{||}$}}1 %needed for correct display in index
              {'a}{{\color{OliveGreen}$\alpha\,$}}1
              {'b}{{\color{OliveGreen}$\beta\,$}}1
              {'c}{{\color{OliveGreen}$\gamma\,$}}1
              {'t}{{\color{OliveGreen}$\tau\,$}}1
              {'x}{{\color{OliveGreen}$\chi\,$}}1
              {lambda}{{$\lambda\,$}}1
              {sumvd}{{$\sum_{\textrm{v} \in \textrm{d}}$}}1
              {o5}{{$\frac{1}{2}$}}1
              {result}{{$\result$}}1 
              {m0}{{$\textrm{m}_0$}}1
              {m1}{{$\textrm{m}_1$}}1
              {l0}{{$\textrm{l}_0$}}1
}

\newcommand{\ec}[1]{\lstinline[mathescape,language=easycrypt,xleftmargin=0pt,xrightmargin=0pt,style=easycrypt-default,basicstyle=\scriptsize\ttfamily,morekeywords={}]{#1}}


\title{\EasyCrypt\ - a (brief) tutorial}
\date{FMiTF Bootcamp - May 29 - June 2, 2023}
\author{Vitor Pereira}

\begin{document}
\maketitle
%\tableofcontents

\section{Introduction}
\label{sec:introduction}

This document serves as support to the \textit{\EasyCrypt - Hashed
  ElGamal semantic security proof exercise}, containing a brief
\EasyCrypt tutorial, describing the most used proof tactics and also
providing an explanation of the \EasyCrypt specification language.

The student is encouraged to follow the default \EasyCrypt manual 
available at \url{https://github.com/EasyCrypt/easycrypt-doc} for a
more complete description on the tool.

\section{Fuctional types, operators and data structures}
\label{sec:func-types-and-ops}
\EasyCrypt's expression language is a higher-order strongly typed
functional language. Following a syntax close to that of ML languages
such as \OCaml of \FSTAR, \EasyCrypt allows a user to define its own
data types and operators as mathematical functions.

\subsection{Types}
\label{sec:basic-types}

\EasyCrypt natively supports the following basic types:

\begin{itemize}
\item \ec{unit} - the (empty) \textit{void} type
\item \ec{bool} - the boolean type
\item \ec{int} - the set of integers
\item \ec{real} - the set of real
\end{itemize}

New types can be defined according to the following syntax, where we
specify a new type that captures tuples of integers and booleans.

\begin{lstlisting}[mathescape,language=easycrypt,xleftmargin=0pt,xrightmargin=0pt,style=easycrypt-default,basicstyle=\scriptsize\ttfamily,morekeywords={}]  
type int_bool_tuple = int * bool.  
\end{lstlisting}

Types can also be left under specified, i.e., without an actual
realization of the type. For example, in the following \EasyCrypt
code, a new type \ec{t} is defined, without provide a concrete value
to it.

\begin{lstlisting}[mathescape,language=easycrypt,xleftmargin=0pt,xrightmargin=0pt,style=easycrypt-default,basicstyle=\scriptsize\ttfamily,morekeywords={}]  
type t. 
\end{lstlisting}

When a type is left under specified, it is common practice to call it
an \textit{abstract type}.

\subsection{Operators}
\label{sec:operators}

Operators in \EasyCrypt are defined using the \ec{op} keyword,
followed by the operator name, arguments, return type and, finally,
the operator body. The \ec{fst} and \ec{snd} operators - that extract
the first and second elements of the \ec{int_bool_type} type,
respectively - can be specified as follows.

\begin{lstlisting}[mathescape,language=easycrypt,xleftmargin=0pt,xrightmargin=0pt,style=easycrypt-default,basicstyle=\scriptsize\ttfamily,morekeywords={}]  
op fst (x : int_bool_tuple) : int = let (a, b) = x in a.
op snd (x : int_bool_tuple) : bool = let (a, b) = x in b.  
\end{lstlisting}

Operators can also be defined for \EasyCrypt native types. For
example, the script

\begin{lstlisting}[mathescape,language=easycrypt,xleftmargin=0pt,xrightmargin=0pt,style=easycrypt-default,basicstyle=\scriptsize\ttfamily,morekeywords={}]  
op double (x : int) : int = x * 2.
\end{lstlisting}

\noindent specifies a function that doubles the value of a given
integer.

The operators described above can be combined with the goal
of defining a function that, taking as input a tuple of integers and
booleans (the \ec{int_bool_tuple} type), returns a new element of the
\ec{int_bool_tuple} type, where the integer value has been doubled.

\begin{lstlisting}[mathescape,language=easycrypt,xleftmargin=0pt,xrightmargin=0pt,style=easycrypt-default,basicstyle=\scriptsize\ttfamily,morekeywords={}]  
op double_int (x : int_bool_tuple) : int_bool_tuple =
  let k = fst x in
  let b = snd b in
  let dk = double k in
  (dk, b).
\end{lstlisting}

Finally, operators can also be left abstract. For example, suppose one
wants to formalize a finite field library in \EasyCrypt. First, one
would need to specify an (abstract) type to capture elements of the finite
field and then define a series of abstract operators that would
specify arithmetic operations.

\begin{lstlisting}[mathescape,language=easycrypt,xleftmargin=0pt,xrightmargin=0pt,style=easycrypt-default,basicstyle=\scriptsize\ttfamily,morekeywords={}]  
op q : int. (* The order of field is a prime q *)

type t. (* Type of elements of the field *)

op fzero : t. (* Zero element *)
op fone  : t. (* One element *)
op ( * ) : t -> t -> t. (* Multiplication modulo q *)
op ( + ) : t -> t -> t. (* Addition modulo q *)
op [ - ] : t -> t. (* Additive inverse modulo q *)
\end{lstlisting}

In the above \EasyCrypt script also introduces the syntax to specify
infix and prefix operators: operators defined between parentheses -
like \ec{op ( * )} and \ec{op ( + )} - are infix operators whereas
operators defined between brackets - like \ec{[ - ]} - are prefix
operators. 

\subsection{Data structures}
\label{sec:data-structures}

\EasyCrypt also supports inductive data structures, commonly found in
many functional languages. Natively, it supports inductive lists that
may be the empty list \ec{[]}, or a value \ec{x::xs} constructed
inductively by prepending \ec{x} to the list {xs}.

\begin{lstlisting}[mathescape,language=easycrypt,xleftmargin=0pt,xrightmargin=0pt,style=easycrypt-default,basicstyle=\scriptsize\ttfamily,morekeywords={}]  
type 'a list = [
  | "[]"
  | (::) of 'a & 'a list
].
\end{lstlisting}

Pattern-matching over an inductive data type is performed using the
\ec{with} syntax, as follows.

\begin{lstlisting}[mathescape,language=easycrypt,xleftmargin=0pt,xrightmargin=0pt,style=easycrypt-default,basicstyle=\scriptsize\ttfamily,morekeywords={}]  
op size (l : 'a list) : int =
  with l = []      => 0
  with l = x :: xs => 1 + size xs.
\end{lstlisting}

% \EasyCrypt already discloses an extremely comprehensive list library,
% that the student is encouraged to 

Another example of an inductive data type that is widely used is that
of binary trees, comprised of \textit{nodes} and \textit{leafs}. A
\textit{node} carries a data item and has a left and right
\textit{subtree}. A \textit{leaf} is empty. In \EasyCrypt, one can
write the binary tree data type as follows.

\begin{lstlisting}[mathescape,language=easycrypt,xleftmargin=0pt,xrightmargin=0pt,style=easycrypt-default,basicstyle=\scriptsize\ttfamily,morekeywords={}]  
type 'a tree =
  | Leaf
  | Node of 'a & 'a tree & 'a tree.
\end{lstlisting}

The depth of a tree is given by the total number of edges from the
root node to the target node. In \EasyCrypt, the depth of a tree can
be calculated as showed next, where we assume the existence of the
function \ec{max} that returns the maximum of two integers.

\begin{lstlisting}[mathescape,language=easycrypt,xleftmargin=0pt,xrightmargin=0pt,style=easycrypt-default,basicstyle=\scriptsize\ttfamily,morekeywords={}]  
op depth (t : 'a tree) : int =
  with t = Leaf       => 0
  with t = Node x l r => 1 + max (depth l) (depth r).
\end{lstlisting}

\section{Ambient logic}
\label{sec:ambient-logic}
\EasyCrypt's is \EasyCrypt's foundational proof engine. It allows
users to write axioms or prove lemmas about existing or newly defined
types and operators. These can be either universally or existentially
quantified.

Before providing examples of \EasyCrypt axioms and lemmas, we first
revise some useful proof tactics.

\subsection{Commonly used \EasyCrypt tactics}
\label{sec:ambient-logic-tactics}
\paragraph{\ec{move =>} $\phi_1 \ldots \phi_n$}

Moves assumptions from the conclusion into the context.

Example: applying \ec{move => x y H} to

\begin{lstlisting}[mathescape,language=easycrypt,xleftmargin=0pt,xrightmargin=0pt,style=easycrypt-default,basicstyle=\scriptsize\ttfamily,morekeywords={}]  
------------------------------------------------------------------------
forall (x y : int), x = y => y = x
\end{lstlisting}

\noindent is transformed into

\begin{lstlisting}[mathescape,language=easycrypt,xleftmargin=0pt,xrightmargin=0pt,style=easycrypt-default,basicstyle=\scriptsize\ttfamily,morekeywords={}]  
x: int
y: int
H: x = y
------------------------------------------------------------------------
y = x
\end{lstlisting}

\paragraph{\ec{move :} $\phi_1 \ldots \phi_n$}

Moves assumptions from the context into the conclusion.

Example: applying \ec{move : x y H} to

\begin{lstlisting}[mathescape,language=easycrypt,xleftmargin=0pt,xrightmargin=0pt,style=easycrypt-default,basicstyle=\scriptsize\ttfamily,morekeywords={}]  
x: int
y: int
H: x = y
------------------------------------------------------------------------
y = x
\end{lstlisting}

\noindent is transformed into

\begin{lstlisting}[mathescape,language=easycrypt,xleftmargin=0pt,xrightmargin=0pt,style=easycrypt-default,basicstyle=\scriptsize\ttfamily,morekeywords={}]  
------------------------------------------------------------------------
forall (x y : int), x = y => y = x
\end{lstlisting}

\paragraph{\ec{split}}

Break a goal whose conclusion is intrinsically conjunctive into goals
whose conclusions are its conjunction.

Example: applying \ec{split} to

\begin{lstlisting}[mathescape,language=easycrypt,xleftmargin=0pt,xrightmargin=0pt,style=easycrypt-default,basicstyle=\scriptsize\ttfamily,morekeywords={}]  
P: bool
Q: bool
Hp: P
Hq: Q
------------------------------------------------------------------------
P /\ Q
\end{lstlisting}

\noindent is transformed into the following subgoals

\begin{lstlisting}[mathescape,language=easycrypt,xleftmargin=0pt,xrightmargin=0pt,style=easycrypt-default,basicstyle=\scriptsize\ttfamily,morekeywords={}]  
P: bool
Q: bool
Hp: P
Hq: Q
------------------------------------------------------------------------
P
\end{lstlisting}

\noindent and

\begin{lstlisting}[mathescape,language=easycrypt,xleftmargin=0pt,xrightmargin=0pt,style=easycrypt-default,basicstyle=\scriptsize\ttfamily,morekeywords={}]  
P: bool
Q: bool
Hp: P
Hq: Q
------------------------------------------------------------------------
Q
\end{lstlisting}

\paragraph{\ec{right}}

Reduce a goal whose conclusion is a disjunction to one whose
conclusion is its right member.

Example: applying \ec{split} to

\begin{lstlisting}[mathescape,language=easycrypt,xleftmargin=0pt,xrightmargin=0pt,style=easycrypt-default,basicstyle=\scriptsize\ttfamily,morekeywords={}]  
P: bool
Q: bool
Hp: P
Hq: Q
------------------------------------------------------------------------
P \/ Q
\end{lstlisting}

\noindent is transformed into

\begin{lstlisting}[mathescape,language=easycrypt,xleftmargin=0pt,xrightmargin=0pt,style=easycrypt-default,basicstyle=\scriptsize\ttfamily,morekeywords={}]  
P: bool
Q: bool
Hp: P
Hq: Q
------------------------------------------------------------------------
Q
\end{lstlisting}

\paragraph{\ec{left}}

Reduce a goal whose conclusion is a disjunction to one whose
conclusion is its left member.

Example: applying \ec{split} to

\begin{lstlisting}[mathescape,language=easycrypt,xleftmargin=0pt,xrightmargin=0pt,style=easycrypt-default,basicstyle=\scriptsize\ttfamily,morekeywords={}]  
P: bool
Q: bool
Hp: P
Hq: Q
------------------------------------------------------------------------
P \/ Q
\end{lstlisting}

\noindent is transformed into

\begin{lstlisting}[mathescape,language=easycrypt,xleftmargin=0pt,xrightmargin=0pt,style=easycrypt-default,basicstyle=\scriptsize\ttfamily,morekeywords={}]  
P: bool
Q: bool
Hp: P
Hq: Q
------------------------------------------------------------------------
P
\end{lstlisting}

\paragraph{\ec{congr}}

Replace a goal whose conclusion has the form $f p_1 \ldots p_n =
f q_1 \ldots q_n$, where \ec{f} is an assumption identifier or
operator, with subgoals having conclusions $p_1 = q_1, \ldots, p_n =
q_n$.

Example: applying \ec{congr} to

\begin{lstlisting}[mathescape,language=easycrypt,xleftmargin=0pt,xrightmargin=0pt,style=easycrypt-default,basicstyle=\scriptsize\ttfamily,morekeywords={}]  
x: int
y: int
a: int
b: int
------------------------------------------------------------------------
f x y = f a b
\end{lstlisting}

\noindent is transformed into the following subgoals

\begin{lstlisting}[mathescape,language=easycrypt,xleftmargin=0pt,xrightmargin=0pt,style=easycrypt-default,basicstyle=\scriptsize\ttfamily,morekeywords={}]  
x: int
y: int
a: int
b: int
------------------------------------------------------------------------
x = a
\end{lstlisting}

\noindent and

\begin{lstlisting}[mathescape,language=easycrypt,xleftmargin=0pt,xrightmargin=0pt,style=easycrypt-default,basicstyle=\scriptsize\ttfamily,morekeywords={}]  
x: int
y: int
a: int
b: int
------------------------------------------------------------------------
y = b
\end{lstlisting}

\paragraph{\ec{trivial}}

Try to solve the goal by using a mixture of low-level tactics.

Example: applying \ec{trivial} to

\begin{lstlisting}[mathescape,language=easycrypt,xleftmargin=0pt,xrightmargin=0pt,style=easycrypt-default,basicstyle=\scriptsize\ttfamily,morekeywords={}]  
------------------------------------------------------------------------
forall (x y : int), x = y => y - 1 = x - 1
\end{lstlisting}

\noindent solves the goal.

\paragraph{\ec{progress}}

Break the goal into multiple simpler ones by repeatedly applying
\ec{move =>}, \ec{split} and \ec{subst}.

Example: applying \ec{progress} to

\begin{lstlisting}[mathescape,language=easycrypt,xleftmargin=0pt,xrightmargin=0pt,style=easycrypt-default,basicstyle=\scriptsize\ttfamily,morekeywords={}]  
------------------------------------------------------------------------
forall (x : 'a) (l : 'a list), x \in l => 1 <= size l
\end{lstlisting}

\noindent is transformed into

\begin{lstlisting}[mathescape,language=easycrypt,xleftmargin=0pt,xrightmargin=0pt,style=easycrypt-default,basicstyle=\scriptsize\ttfamily,morekeywords={}]
x: 'a
l: 'a list
H: x \in l
------------------------------------------------------------------------
1 <= size l
\end{lstlisting}

\paragraph{\ec{have :} $\phi$}

Logical cut. Generate two subgoals: one whose conclusion is the cut
formula $\phi$, and one with conclusion $\phi => \psi$, where $\psi$
is the current goal’s conclusion.

Example: applying \ec{have : x \/ (x => false)} to

\begin{lstlisting}[mathescape,language=easycrypt,xleftmargin=0pt,xrightmargin=0pt,style=easycrypt-default,basicstyle=\scriptsize\ttfamily,morekeywords={}]
x: bool
notnot_x: (x => false) => false
------------------------------------------------------------------------
x
\end{lstlisting}

\noindent is transformed into the following subgoals

\begin{lstlisting}[mathescape,language=easycrypt,xleftmargin=0pt,xrightmargin=0pt,style=easycrypt-default,basicstyle=\scriptsize\ttfamily,morekeywords={}]
x: bool
notnot_x: (x => false) => false
------------------------------------------------------------------------
x \/ (x => false)
\end{lstlisting}

\noindent and

\begin{lstlisting}[mathescape,language=easycrypt,xleftmargin=0pt,xrightmargin=0pt,style=easycrypt-default,basicstyle=\scriptsize\ttfamily,morekeywords={}]
x: bool
notnot_x: (x => false) => false
------------------------------------------------------------------------
x \/ (x => false) => x
\end{lstlisting}

\paragraph{\ec{apply} $\phi$}

Tries to match the conclusion of the proof term $\phi$ with the goal’s
conclusion.

Example: applying \ec{apply H} to

\begin{lstlisting}[mathescape,language=easycrypt,xleftmargin=0pt,xrightmargin=0pt,style=easycrypt-default,basicstyle=\scriptsize\ttfamily,morekeywords={}]
x: int
H: P x  
------------------------------------------------------------------------
P x
\end{lstlisting}

\noindent solves the goal.

\paragraph{\ec{rewrite} $\phi_1 \ldots \phi_n$}

Rewrite the rewrite-pattern $\phi_1 \ldots \phi_n$ from left to right.

Example: applying \ec{rewrite eq_xy} to

\begin{lstlisting}[mathescape,language=easycrypt,xleftmargin=0pt,xrightmargin=0pt,style=easycrypt-default,basicstyle=\scriptsize\ttfamily,morekeywords={}]
x: int
y: int  
eq_xy: x = y
z: int
eq_yz: y = z
------------------------------------------------------------------------
x = z
\end{lstlisting}

\noindent is transformed into

\begin{lstlisting}[mathescape,language=easycrypt,xleftmargin=0pt,xrightmargin=0pt,style=easycrypt-default,basicstyle=\scriptsize\ttfamily,morekeywords={}]
x: int
y: int  
eq_xy: x = y
z: int
eq_yz: y = z
------------------------------------------------------------------------
y = z
\end{lstlisting}

\paragraph{\ec{subst} $\phi$}

Search for the first equation of the form $x = t$ or $t = x$ in the
context and replace all the occurrences of x by t everywhere in the
context and the conclusion before clearing it.

Example: applying \ec{subst x} to

\begin{lstlisting}[mathescape,language=easycrypt,xleftmargin=0pt,xrightmargin=0pt,style=easycrypt-default,basicstyle=\scriptsize\ttfamily,morekeywords={}]
x: bool
y: bool
z: bool
w: bool
eq_yx: y = x
eq_yz: y = z
eq_zw: z = w
------------------------------------------------------------------------
x = w
\end{lstlisting}

\noindent is transformed into

\begin{lstlisting}[mathescape,language=easycrypt,xleftmargin=0pt,xrightmargin=0pt,style=easycrypt-default,basicstyle=\scriptsize\ttfamily,morekeywords={}]
x: bool
y: bool
z: bool
w: bool
eq_yz: y = z
eq_zw: z = w
------------------------------------------------------------------------
y = w
\end{lstlisting}

\paragraph{\ec{case} $\phi$}

Assuming the goal’s conclusion is not a statement judgement, do an
excluded-middle case analysis on $\phi$, substituting $\phi$ in the goal’s
conclusion.

Example: applying \ec{case (x <= y)} to

\begin{lstlisting}[mathescape,language=easycrypt,xleftmargin=0pt,xrightmargin=0pt,style=easycrypt-default,basicstyle=\scriptsize\ttfamily,morekeywords={}]
x: int
y: int
------------------------------------------------------------------------
0 <= y - x
\end{lstlisting}

\noindent is transformed into the following subgoals

\begin{lstlisting}[mathescape,language=easycrypt,xleftmargin=0pt,xrightmargin=0pt,style=easycrypt-default,basicstyle=\scriptsize\ttfamily,morekeywords={}]
x: int
y: int  
------------------------------------------------------------------------
x <= y
\end{lstlisting}

\noindent and

\begin{lstlisting}[mathescape,language=easycrypt,xleftmargin=0pt,xrightmargin=0pt,style=easycrypt-default,basicstyle=\scriptsize\ttfamily,morekeywords={}]
x: int
y: int
------------------------------------------------------------------------
x <= y => 0 <= y - x
\end{lstlisting}

\paragraph{\ec{elim} $\phi$}

Eliminates the top assumption of the goal’s conclusion, generating
subgoals that are dependent upon the kind of assumption eliminated.

Example: applying \ec{elim l} to

\begin{lstlisting}[mathescape,language=easycrypt,xleftmargin=0pt,xrightmargin=0pt,style=easycrypt-default,basicstyle=\scriptsize\ttfamily,morekeywords={}]
l: 'a list
------------------------------------------------------------------------
0 <= size l
\end{lstlisting}

\noindent is transformed into the following subgoals

\begin{lstlisting}[mathescape,language=easycrypt,xleftmargin=0pt,xrightmargin=0pt,style=easycrypt-default,basicstyle=\scriptsize\ttfamily,morekeywords={}]
------------------------------------------------------------------------
0 <= size []
\end{lstlisting}

\noindent and

\begin{lstlisting}[mathescape,language=easycrypt,xleftmargin=0pt,xrightmargin=0pt,style=easycrypt-default,basicstyle=\scriptsize\ttfamily,morekeywords={}]
------------------------------------------------------------------------
forall (x : 'a) (l : 'a list), 0 <= size l => 0 <= size (x :: l)
\end{lstlisting}

\paragraph{\ec{simplify}}

Attempts to simplify the proof goal by solving trivial equalities or
even by expanding operators being used.

Example: applying \ec{simplify} to

\begin{lstlisting}[mathescape,language=easycrypt,xleftmargin=0pt,xrightmargin=0pt,style=easycrypt-default,basicstyle=\scriptsize\ttfamily,morekeywords={}]
x: 'a
l: 'a list
H: 0 <= size l
------------------------------------------------------------------------
0 <= size (x :: l)
\end{lstlisting}

\noindent is transformed into

\begin{lstlisting}[mathescape,language=easycrypt,xleftmargin=0pt,xrightmargin=0pt,style=easycrypt-default,basicstyle=\scriptsize\ttfamily,morekeywords={}]
x: 'a
l: 'a list
H: 0 <= size l
------------------------------------------------------------------------
0 <= 1 + size l
\end{lstlisting}

\paragraph{\ec{assumption}}

Search in the context for a hypothesis that is convertible to the
goal’s conclusion, solving the goal if one is found. Fail if none can
be found.

Example: applying \ec{assumption} to

\begin{lstlisting}[mathescape,language=easycrypt,xleftmargin=0pt,xrightmargin=0pt,style=easycrypt-default,basicstyle=\scriptsize\ttfamily,morekeywords={}]
x: bool
H: P x  
------------------------------------------------------------------------
P x
\end{lstlisting}

\noindent solves the goal.

\paragraph{\ec{reflexivity}}

Solve goals with conclusions of the form $x = x$ (up to computation).

Example: applying \ec{reflexivity} to

\begin{lstlisting}[mathescape,language=easycrypt,xleftmargin=0pt,xrightmargin=0pt,style=easycrypt-default,basicstyle=\scriptsize\ttfamily,morekeywords={}]
x: bool
------------------------------------------------------------------------
x = x
\end{lstlisting}

\noindent solves the goal.

\paragraph{\ec{done}}

Apply \ec{trivial} and fail if the goal is not closed.

\paragraph{\ec{smt}}

Try to solve the goal using SMT solvers. The goal is sent along with
the local hypotheses plus selected axioms and lemmas.

Example: applying \ec{smt} to

\begin{lstlisting}[mathescape,language=easycrypt,xleftmargin=0pt,xrightmargin=0pt,style=easycrypt-default,basicstyle=\scriptsize\ttfamily,morekeywords={}]
x: int
y: int
z: int
H: x = y
H0: y = z
------------------------------------------------------------------------
x = z
\end{lstlisting}

\noindent solves the goal.

\subsection{Axioms}
\label{sec:axioms}

\EasyCrypt's allows users to axiomatize properties regarding types and
operators. For example, using the (small) finite field library defined
in Section~\ref{sec:operators}, it is possible to formalize the
expected properties of the field operators, like the commutativity or
associative properties, using axioms as follows.

\begin{lstlisting}[mathescape,language=easycrypt,xleftmargin=0pt,xrightmargin=0pt,style=easycrypt-default,basicstyle=\scriptsize\ttfamily,morekeywords={}]
axiom addC (x y : t): x + y = y + x. (* Commutative addition property *)
axiom addA (x y z : t) : x + (y + z) = (x + y) + z. (* Associative addition property *)

axiom mulC (x y : t) : x * y = y * x. (* Commutative addition property *)
axiom mulA (x y z : t): x * (y * z) = (x * y) * z. (* Associative multiplication property *)
axiom mulfDl (x y z : t): (x + y) * (x + z) = x * (y + z). (* Distributive multiplication property over the addition *)
\end{lstlisting}

\subsection{Lemmas}
\label{sec:lemmas}

Lemmas are properties that, unlike axioms, are not assumed to be true
and that require the user to write a complete proof for it. We provide
two examples of \EasyCrypt lemmas, together with their respective
proof script:

\begin{enumerate}
\item one that proves that adding a field element to another element that is different than
  zero will output a different field element; and
\item one that proves that the size of any list is always greater
  than or equal to zero.
\end{enumerate}

\subsubsection{Adding a field element to another field element
  different than zero}

To prove the desired property, one can write the following two
lemmas. It is recommended to follow this example using the \EasyCrypt
framework in order to get a clear picture of how the proof evolves.

\begin{lstlisting}[mathescape,language=easycrypt,xleftmargin=0pt,xrightmargin=0pt,style=easycrypt-default,basicstyle=\scriptsize\ttfamily,morekeywords={}]
lemma add_fzero_imp (x : t) (y : t) : x + y = x => y = fzero.
proof.
  move => H. 
  have : y = x + (- x) by smt.
  move => H0.
  rewrite H0. 
  apply addfN. 
qed.
  
lemma non_zero_add (x : t) (y : t) :
  y <> fzero => x + y <> x.
proof.
  move => H.
  case (x + y = x).
    move => H2.
    have : y = fzero.
      rewrite (add_fzero_imp x y).
        assumption.
      reflexivity.
    trivial.
  trivial.
qed.
\end{lstlisting}

\subsubsection{Size of any list is always greater than or equal to
  zero}

To prove the desired property, one can write the following lemma. It
is recommended to follow this example using the \EasyCrypt
framework in order to get a clear picture of how the proof evolves.

\begin{lstlisting}[mathescape,language=easycrypt,xleftmargin=0pt,xrightmargin=0pt,style=easycrypt-default,basicstyle=\scriptsize\ttfamily,morekeywords={}]
lemma size_ge0 (l : 'a list) : 0 <= size l.
proof.
  elim l.
    simplify.
    trivial.
  move => x l Hind.
  simplify.
  smt.
qed.
\end{lstlisting}

\section{Modules}
\label{sec:modules}
So far, we have explored the \textit{functional} core of
\EasyCrypt. Complementary to it, \EasyCrypt also discloses an
\textit{imperative} subset, captured by \textit{modules}.

\EasyCrypt features a module system that provides a structuring
mechanism for describing imperative constructions. Modules are
composed of a \textit{memory} (a set of global variables, here empty)
and a set of procedures. Procedures in the same module may share state; it is
therefore not necessary to explicitly add state to the module
signature. In addition, modules can be parameterised by other modules
(in which case, we often call them \textit{functors}) whose procedures they can
query like oracles.

Modules are mainly used for representing cryptographic games - either
concrete or abstract. It uses a simple \textit{while} language. For
example, the \INDCPA security game can be is
represented as the following concrete module:

\begin{lstlisting}[mathescape,language=easycrypt,xleftmargin=0pt,xrightmargin=0pt,style=easycrypt-default,basicstyle=\scriptsize\ttfamily,morekeywords={}]  
module INDCPA (S: Scheme) (A: Adversary) = {
  proc main() : bool = {
    var b, b', m0, m1, k, m;

    k <@ S.key_gen();
    (m0, m1) <@ A.gen_query();
    b <$\$$ {0,1};
    m <- if b then m1 else m0;
    c <@ S.encrypt(k, m);
    b' <@ A.guess(c);

    return b';
  }
}.
\end{lstlisting}

In this module, the secret key is first generated by accessing the
\ec{key_gen} procedure. Then, the \textit{adversary} selects two
messages \ec{m0} and \ec{m1}. The game proceeds by randomly sampling a
bit that is used to determine which message is going to be
encrypted. Finally, the adversary, will try to determine if the ciphertext
given to it came from an encryption of \ec{m0} or \ec{m1}.

Note that we make use of different assignment syntaxes:
\begin{itemize}
\item \ec{<-} - assignment of an expression
\item \ec{<@} - assignment of the output of a function call
\item \ec{<}$\$$ - random assignment, i.e., a random value will be
  sampled from a probability distribution
\end{itemize}

The \ec{INDCPA} module is parameterized by a module of \textit{type}
\ec{Scheme} and another of \textit{type} \ec{Adversary}.
%
The constituents of a module and their types are reflected in their
\textit{module type}: a module \ec{M} has module type \ec{I} if all procedures declared
in \ec{I} are also defined in \ec{M}, with the same type and parameters. For
instance, the \ec{Scheme} \textit{module type}, intended to capture
the \textit{type} of symmetric encryption schemes, can be defined as
follows

\begin{lstlisting}[mathescape,language=easycrypt,xleftmargin=0pt,xrightmargin=0pt,style=easycrypt-default,basicstyle=\scriptsize\ttfamily,morekeywords={}]  
module type Scheme = {
  module key_gen() : key
  module encrypt(k : key, pt : plaintext) : ciphertext
  module decrypt(k : key, ct : ciphertext) : plaintext
}
\end{lstlisting}

\noindent meaning that a module that follows this interface will be
considered to have \textit{type} \ec{Scheme}.

\section{Hoare logic}
\label{sec:hoare-logic}
To deconstruct imperative programs, \EasyCrypt incorporates a Hoare
logic proof engine. In \EasyCrypt, a Hoare triple can be written
according to the following syntax

\begin{lstlisting}[mathescape,language=easycrypt,xleftmargin=0pt,xrightmargin=0pt,style=easycrypt-default,basicstyle=\scriptsize\ttfamily,morekeywords={}]  
lemma hoare_triple : hoare [p : pre ==> post ]
\end{lstlisting}

\noindent where \ec{p} is the procedure to be analyzed, \ec{pre} is
the precondition and \ec{post} is the postcondition.

\subsection{Commonly used Hoare logic tactics}
\label{sec:hoare-tactics}

\paragraph{\ec{proc}} Turn a goal whose conclusion is a Hoare logic
judgement involving concrete procedure(s) into one whose conclusion
is a statement judgement by replacing the concrete procedure(s) by
their body/ies.

Example: applying \ec{proc} to

\begin{lstlisting}[mathescape,language=easycrypt,xleftmargin=0pt,xrightmargin=0pt,style=easycrypt-default,basicstyle=\scriptsize\ttfamily,morekeywords={}]  
M : M
------------------------------------------------------------------------
pre = true

    Example(M).main

post = true
\end{lstlisting}

\noindent is transformed into

\begin{lstlisting}[mathescape,language=easycrypt,xleftmargin=0pt,xrightmargin=0pt,style=easycrypt-default,basicstyle=\scriptsize\ttfamily,morekeywords={}]  
M : M
------------------------------------------------------------------------
Context : {x, r, y, z : int}

pre = true

(1--)  z <@ M.gen()             
(2--)  r <$\$$ [0..100]            
(3--)  if (x < 100) {           
(3.1)    y <- x * 2             
(3--)  } else {                 
(3?1)    y <- r + z             
(3--)  }                        

post = true
\end{lstlisting}

\noindent where \ec{r <}$\$$ \ec{[0..100]} captures the integer random
sampling in the $[0;100]$ range.

\paragraph{\ec{wp}} Applies the weakest precondition calculus strategy
to the current program. \ec{wp} will consume assignments, as well as
\textit{if} conditionals whose body does not encompass any random
sample or function calls.

Example: applying \ec{wp} to

\begin{lstlisting}[mathescape,language=easycrypt,xleftmargin=0pt,xrightmargin=0pt,style=easycrypt-default,basicstyle=\scriptsize\ttfamily,morekeywords={}]  
M : M
------------------------------------------------------------------------
Context : {x, r, y, z : int}

pre = true

(1--)  z <@ M.gen()             
(2--)  r <$\$$ [0..100]            
(3--)  if (x < 100) {           
(3.1)    y <- x * 2             
(3--)  } else {                 
(3?1)    y <- r + z             
(3--)  }                        

post = true
\end{lstlisting}

\noindent is transformed into

\begin{lstlisting}[mathescape,language=easycrypt,xleftmargin=0pt,xrightmargin=0pt,style=easycrypt-default,basicstyle=\scriptsize\ttfamily,morekeywords={}]
M : M
------------------------------------------------------------------------
Context : {x, r, y, z : int}

pre = true

(1)  z <@ M.gen()             
(2)  r <$\$$ [0..100]            

post = if x < 100 then true else true
\end{lstlisting}

\paragraph{\ec{rnd}} If the conclusion is a Hoare logic judgment
whose program ends with a random assignments \ec{x <}$\$$ \ec{d}, then consume
those random assignments, replacing the conclusion’s postcondition by
the probabilistic weakest precondition of the random assignments.

Example: applying \ec{rnd} to

\begin{lstlisting}[mathescape,language=easycrypt,xleftmargin=0pt,xrightmargin=0pt,style=easycrypt-default,basicstyle=\scriptsize\ttfamily,morekeywords={}]  
M : M
------------------------------------------------------------------------
Context : {x, r, y, z : int}

pre = true

(1)  z <@ M.gen()             
(2)  r <$\$$ [0..100]            

post = if x < 100 then true else true
\end{lstlisting}

\noindent is transformed into

\begin{lstlisting}[mathescape,language=easycrypt,xleftmargin=0pt,xrightmargin=0pt,style=easycrypt-default,basicstyle=\scriptsize\ttfamily,morekeywords={}]  
M : M
------------------------------------------------------------------------
Context : {x, r, y, z : int}

pre = true

(1)  z <@ M.gen()             

post =
  forall (r0 : int),
    (r0 \in [0..100])%Distr => if x < 100 then true else true
\end{lstlisting}

\paragraph{\ec{call (_ : )}$\phi$}

If the conclusion is a Hoare logic judgement whose program end
with a function call of the same abstract procedure,
then use the specification argument to call generated from the
invariant $\phi$, and automatically apply \ec{proc} $\phi$ to its first subgoal,
pruning the first two subgoals the application generates, because
their conclusions consist of ambient logic formulas that are true by
construction.

Example: applying \ec{call _ : true} to

\begin{lstlisting}[mathescape,language=easycrypt,xleftmargin=0pt,xrightmargin=0pt,style=easycrypt-default,basicstyle=\scriptsize\ttfamily,morekeywords={}]  
M : M
------------------------------------------------------------------------
Context : {x, r, y, z : int}

pre = true

(1)  z <@ M.gen()             

post =
  forall (r0 : int),
    (r0 \in [0..100])%Distr => if x < 100 then true else true
\end{lstlisting}

\noindent is transformed to

\begin{lstlisting}[mathescape,language=easycrypt,xleftmargin=0pt,xrightmargin=0pt,style=easycrypt-default,basicstyle=\scriptsize\ttfamily,morekeywords={}]  
M : M
------------------------------------------------------------------------
Context : {x, r, y, z : int}

pre = true


post =
  forall (r0 : int),
    (r0 \in [0..100])%Distr => if x < 100 then true else true
\end{lstlisting}

\paragraph{\ec{skip}}

If the goal’s conclusion is a statement judgement whose program(s) are
empty, reduce it to the goal whose conclusion is the ambient logic
formula $\phi => \psi$, where $\phi$ is the original conclusion’s precondition, and
$\psi$ is its postcondition.

Example: applying \ec{skip} to

\begin{lstlisting}[mathescape,language=easycrypt,xleftmargin=0pt,xrightmargin=0pt,style=easycrypt-default,basicstyle=\scriptsize\ttfamily,morekeywords={}]  
M : M
------------------------------------------------------------------------
Context : {x, r, y, z : int}

pre = true


post =
  forall (r0 : int),
    (r0 \in [0..100])%Distr => if x < 100 then true else true
\end{lstlisting}

\noindent is transformed to

\begin{lstlisting}[mathescape,language=easycrypt,xleftmargin=0pt,xrightmargin=0pt,style=easycrypt-default,basicstyle=\scriptsize\ttfamily,morekeywords={}]  
M : M
------------------------------------------------------------------------
forall &hr,
  true =>
  forall (r0 : int),
    (r0 \in [0..100])%Distr => if x{hr} < 100 then true else true
\end{lstlisting}

\paragraph{\ec{while} $\phi$}

If the goal’s conclusion is a Hoare logic judgement whose program
ends with a \ec{while} statements, reduce the goal to two subgoals whose
conclusions are Hoare logic judgments:

\begin{itemize}
\item One whose program is the body of the \ec{while} statement, whose
  precondition is the conjunction of $\phi$ and the while statements’
  boolean expressions and whose postcondition is the conjunction of
  $\phi$ and the assertion that the while statements’ boolean
  expressions (interpreted in the appropriate memories) are
  equivalent. Essentially, one is required to prove that the invariant
  $\phi$ is preserved throughout the loop execution
\item One whose precondition is the original goal’s precondition,
  whose program is the results of removing the while statement from
  the original program, and whose postcondition is the conjunction of:
  \begin{itemize}
  \item the conjunction of $\phi$ and the assertion that the while
    statement’s boolean expressions are equivalent; and
  \item the assertion that, for all values of the variables modified
    by the while statement, if the while statement’s boolean
    expressions don’t hold, but $\phi$ holds, then the original goal’s
    postcondition holds.
  \end{itemize}

  Essentially, one is required to prove that the invariant holds at
  the beginning of the loop and at the end of the loop.
\end{itemize}

Example: applying \ec{while (0 <= i <= 10 /\ y = x * i)} to

\begin{lstlisting}[mathescape,language=easycrypt,xleftmargin=0pt,xrightmargin=0pt,style=easycrypt-default,basicstyle=\scriptsize\ttfamily,morekeywords={}]  
_x: int
------------------------------------------------------------------------
Context : {x, y, i : int}

pre = x = _x

(1--)  i <- 0                   
(2--)  y <- 0                   
(3--)  while (i < 10) {         
(3.1)    y <- y + x             
(3.2)    i <- i + 1             
(3--)  }                        

post = y = _x * 10
\end{lstlisting}

\noindent is transformed into the following subgoals

\begin{lstlisting}[mathescape,language=easycrypt,xleftmargin=0pt,xrightmargin=0pt,style=easycrypt-default,basicstyle=\scriptsize\ttfamily,morekeywords={}]  
_x: int
------------------------------------------------------------------------
Context : {x, y, i : int}

pre = ((0 <= i && i <= 10) /\ y = x * i) /\ i < 10

(1)  y <- y + x               
(2)  i <- i + 1               

post = (0 <= i && i <= 10) /\ y = x * i
\end{lstlisting}

\noindent and

\begin{lstlisting}[mathescape,language=easycrypt,xleftmargin=0pt,xrightmargin=0pt,style=easycrypt-default,basicstyle=\scriptsize\ttfamily,morekeywords={}]  
_x: int
------------------------------------------------------------------------
Context : {x, y, i : int}

pre = x = _x

(1)  i <- 0                   
(2)  y <- 0                   

post =
  ((0 <= i && i <= 10) /\ y = x * i) /\
  forall (i0 y0 : int),
    ! i0 < 10 => (0 <= i0 && i0 <= 10) /\ y0 = x * i0 => y0 = _x * 10
\end{lstlisting}

\paragraph{\ec{if}}

If the goal’s conclusion is a Hoare logic judgement whose program begin
with an if statement, reduces the goal to two subgoals:
\begin{itemize}
\item One in which the if statement has been replaced by its
  \textit{then} branch, and where the assertion of the truth of the
  if statement’s boolean expression has   been added to the
  conclusion’s precondition.
\item One in which the if statement has been replaced by its
  \textit{else} part, and where the assertion of the falsity of the if
  statement’s boolean expression has
  been added to the conclusion’s precondition.
\end{itemize}

Example: applying \ec{if} to

\begin{lstlisting}[mathescape,language=easycrypt,xleftmargin=0pt,xrightmargin=0pt,style=easycrypt-default,basicstyle=\scriptsize\ttfamily,morekeywords={}]  
_x: int
------------------------------------------------------------------------
Context : {x, y : int}

pre = x = _x /\ x < 100

(1--)  if (x < 100) {           
(1.1)    y <- x                 
(1--)  } else {                 
(1?1)    y <- x * 2             
(1--)  }                        

post = y = _x
\end{lstlisting}

\noindent is transformed into the following subgoals

\begin{lstlisting}[mathescape,language=easycrypt,xleftmargin=0pt,xrightmargin=0pt,style=easycrypt-default,basicstyle=\scriptsize\ttfamily,morekeywords={}]  
_x: int
------------------------------------------------------------------------
Context : {x, y : int}

pre = (x = _x /\ x < 100) /\ x < 100

(1)  y <- x                   

post = y = _x
\end{lstlisting}

\noindent and

\begin{lstlisting}[mathescape,language=easycrypt,xleftmargin=0pt,xrightmargin=0pt,style=easycrypt-default,basicstyle=\scriptsize\ttfamily,morekeywords={}]  
_x: int
------------------------------------------------------------------------
Context : {x, y : int}

pre = (x = _x /\ x < 100) /\ ! x < 100

(1)  y <- x * 2               

post = y = _x
\end{lstlisting}

\paragraph{\ec{rcondt n}}

If the goal’s conclusion is an Hoare logic judgement whose $n$
statement is an \ec{if} statement, reduce the goal to two subgoals:

\begin{itemize}
\item One whose concludion is an Hoare logic judgement whose
  precondition is the original goal’s precondition, whose program is the
  first $n-1$ statements of the original goal’s program, and
  whose postcondition is the boolean expression of the if statement.
\item One whose conclusion is an Hoare logic judgement that’s the
  same as that of the original goal except that the \ec{if} statement has
  been replaced by its then part.
\end{itemize}

Example: applying \ec{rcondt 1} to

\begin{lstlisting}[mathescape,language=easycrypt,xleftmargin=0pt,xrightmargin=0pt,style=easycrypt-default,basicstyle=\scriptsize\ttfamily,morekeywords={}]  
_x: int
------------------------------------------------------------------------
Context : {x, y : int}

pre = x = _x /\ x < 100

(1--)  if (x < 100) {           
(1.1)    y <- x                 
(1--)  } else {                 
(1?1)    y <- x * 2             
(1--)  }                        

post = y = _x
\end{lstlisting}

\noindent is transformed into the following subgoals

\begin{lstlisting}[mathescape,language=easycrypt,xleftmargin=0pt,xrightmargin=0pt,style=easycrypt-default,basicstyle=\scriptsize\ttfamily,morekeywords={}]  
_x: int
------------------------------------------------------------------------
Context : {x, y : int}

pre = x = _x /\ x < 100


post = x < 100
\end{lstlisting}

\noindent and

\begin{lstlisting}[mathescape,language=easycrypt,xleftmargin=0pt,xrightmargin=0pt,style=easycrypt-default,basicstyle=\scriptsize\ttfamily,morekeywords={}]  
_x: int
------------------------------------------------------------------------
Context : {x, y : int}

pre = x = _x /\ x < 100

(1)  y <- x                   

post = y = _x
\end{lstlisting}

\paragraph{\ec{rcondf n}}

If the goal’s conclusion is an HL statement judgement whose $n$
statement is an \ec{if} statement, reduce the goal to two subgoals:

\begin{itemize}
\item One whose concludion is an Hoare logic judgment whose
  precondition is the original goal’s precondition, whose program is the
  first $n-1$ statements of the original goal’s program, and
  whose postcondition is the negation of the boolean expression of the if
  statement.
\item One whose conclusion is an Hoare logic judgement that’s the
  same as that of the original goal except that the \ec{if} statement has
  been replaced by its else part. 
\end{itemize}

Example: applying \ec{rcondf 1} to

\begin{lstlisting}[mathescape,language=easycrypt,xleftmargin=0pt,xrightmargin=0pt,style=easycrypt-default,basicstyle=\scriptsize\ttfamily,morekeywords={}]  
_x: int
------------------------------------------------------------------------
Context : {x, y : int}

pre = x = _x /\ x < 100

(1--)  if (x < 100) {           
(1.1)    y <- x                 
(1--)  } else {                 
(1?1)    y <- x * 2             
(1--)  }                        

post = y = _x
\end{lstlisting}

\noindent is transformed into the following subgoals

\begin{lstlisting}[mathescape,language=easycrypt,xleftmargin=0pt,xrightmargin=0pt,style=easycrypt-default,basicstyle=\scriptsize\ttfamily,morekeywords={}]  
_x: int
------------------------------------------------------------------------
Context : {x, y : int}

pre = x = _x /\ x < 100


post = ! x < 100
\end{lstlisting}

\noindent and

\begin{lstlisting}[mathescape,language=easycrypt,xleftmargin=0pt,xrightmargin=0pt,style=easycrypt-default,basicstyle=\scriptsize\ttfamily,morekeywords={}]  
_x: int
------------------------------------------------------------------------
Context : {x, y : int}

pre = x = _x /\ x < 100

(1)  y <- x * 2               

post = y = _x
\end{lstlisting}

\subsection{\EasyCrypt Hoare logic example}
\label{sec:hoare-logic-example}

Consider the following \EasyCrypt module

\begin{lstlisting}[mathescape,language=easycrypt,xleftmargin=0pt,xrightmargin=0pt,style=easycrypt-default,basicstyle=\scriptsize\ttfamily,morekeywords={}]
module type M = {
  proc gen() : int
}.

module Example (M : M) = {
  
  proc main(x : int) : int = {
    var r, y, z;

    z <@ M.gen();
    r <$\$$ [0..100];
    if (x < 100) { y <- x*2; }
    else { y <- r + z; }

    return y;
  }
}.
\end{lstlisting}

\noindent where we define a module \ec{Example} with a procedure
\ec{main} that either doubles its input or that assigns it to the
output of a procedure call added to a random value. In this example,
we will prove that if the input value \ec{x} is less than 100
(precondition), then the output will be \ec{x*2} (postcondition).

The \EasyCrypt Hoare triple lemma is written bellow.

\begin{lstlisting}[mathescape,language=easycrypt,xleftmargin=0pt,xrightmargin=0pt,style=easycrypt-default,basicstyle=\scriptsize\ttfamily,morekeywords={}]
lemma example (M <: M) (_x : int) : hoare [Example(M).main : _x = x /\ x < 100 ==> res = _x * 2].
\end{lstlisting}

The \ec{example} lemma does two universal quantifications:
%
\begin{inparaenum}[i.]
\item one over every possible modules of type \ec{M} (using the
  \ec{<:} notation); and
\item one over every possible integer \ec{_x}
\end{inparaenum}
%
While the former is done to correctly instantiate the \ec{Example}
module, the latter is done to be able to refer to the value of \ec{x}
before the program is executed. This allows us to store the value of
\ec{x} at the beginning of the evaluation and refer to it at the
postcondition. Note also that the postcondition uses a special value
dubbed \ec{res}. This is an \EasyCrypt keyword used to refer to the
output of the program.

The following proof script is able to discharge the afore mentioned
Hoare triple.

\begin{lstlisting}[mathescape,language=easycrypt,xleftmargin=0pt,xrightmargin=0pt,style=easycrypt-default,basicstyle=\scriptsize\ttfamily,morekeywords={}]
lemma example (M <: M) (_x : int) : hoare [Example(M).main : _x = x /\ x < 100 ==> res = _x * 2].
proof.
  proc.
  wp.
  rnd.
  call (_ : true).
  skip.
  move => &hr H result r H0.
  have : x{hr} < 100 by smt().
  have : x{hr} = _x by smt().
  move => H1 H2.
  rewrite H2.
  simplify.
  rewrite H1.
  reflexivity.
qed.
\end{lstlisting}

Again, for a better understanding of the proof process, it is highly
recommended to reproduce the proof script in \EasyCrypt.

\section{Probabilistic Hoare logic}
\label{sec:phoare-logic}
Probabilistic Hoare logic (pHL) allows one to write HL lemmas that are
bounded by some probability. Intuitively, it allows the proof of
statements where, given some precondition, the postcondition only
occurs with a given probability $P$. A pHL triple can be written
according to the following syntax

\begin{lstlisting}[mathescape,language=easycrypt,xleftmargin=0pt,xrightmargin=0pt,style=easycrypt-default,basicstyle=\scriptsize\ttfamily,morekeywords={}]  
lemma phoare_triple : phoare [p : pre ==> post ] < P
\end{lstlisting}

Dealing with probability distributions inside \EasyCrypt is, perhaps,
the most complicated aspect of \EasyCrypt. \EasyCrypt provides a
series of libraries to deal with probabilistic reasoning that we will
not cover here. Instead, we will resort to the most relevant aspects
of how probability distributions are formalized in \EasyCrypt.

\subsection{\EasyCrypt probability distributions}
\label{sec:probability-distributions}

Probability distributions in \EasyCrypt are defined using the special
\ec{distr} type. For example, a probability distribution over the
integers can be defined as

\begin{lstlisting}[mathescape,language=easycrypt,xleftmargin=0pt,xrightmargin=0pt,style=easycrypt-default,basicstyle=\scriptsize\ttfamily,morekeywords={}]  
op int_distr : int distr.
\end{lstlisting}

The \ec{support} of a distribution represents the elements that
compose the domain of that distribution, i.e., those than can be
sampled. For example, to restrict the domain of \ec{int_distr} to the
values between 0 and 100, one can write

\begin{lstlisting}[mathescape,language=easycrypt,xleftmargin=0pt,xrightmargin=0pt,style=easycrypt-default,basicstyle=\scriptsize\ttfamily,morekeywords={}]  
axiom int_distr_support : forall (x : int), 0 <= x <= 100 => x \in int_distr.
axiom int_distr_supportN : forall (x : int), !(0 <= x <= 100) => x \notin int_distr.
\end{lstlisting}

The \ec{weight} of a distribution establishes the sum of the
probabilities of all elements of the distribution domains. Informally,
we say that if the weight of a distribution is 1, then it is defined
for all elements of the domain. In \EasyCrypt, this is captured by the
\ec{is_lossless} predicate.

\begin{lstlisting}[mathescape,language=easycrypt,xleftmargin=0pt,xrightmargin=0pt,style=easycrypt-default,basicstyle=\scriptsize\ttfamily,morekeywords={}]  
axiom int_distr_lossless : is_lossless int_distr.
\end{lstlisting}

Finally, it is possible to specify the probability of sampling a value
in a probability distribution. In order to do so, \EasyCrypt includes
a special operator \ec{mu}, that defines the probability of some event
occurs in a distribution. Therefore, to define the sampling
probability of an element, one can follow the next \EasyCrypt script.

\begin{lstlisting}[mathescape,language=easycrypt,xleftmargin=0pt,xrightmargin=0pt,style=easycrypt-default,basicstyle=\scriptsize\ttfamily,morekeywords={}]  
axiom int_distr_mu1 : forall (x : int), 0 <= x <= 100 => mu int_distr (fun k => k = x) = (1%r / 101%r).
\end{lstlisting}

\subsection{\EasyCrypt pHL example}
\label{sec:phl-example}

Consider the following \EasyCrypt module, that uses the previously
formalized \ec{int_distr} probability distribution.

\begin{lstlisting}[mathescape,language=easycrypt,xleftmargin=0pt,xrightmargin=0pt,style=easycrypt-default,basicstyle=\scriptsize\ttfamily,morekeywords={}]  
module Example = {
  
  proc main() : int = {
    var x;

    x <$\$$ int_distr;

    return (x * 2);
  }
}.
\end{lstlisting}

\noindent where we define a module \ec{Example} that samples a value
from \ec{int_distr} and then doubles it. In this example, we will
prove that the probability of this program outputting 100 is
$\frac{1}{101}$, i.e., the probability of sampling 50.

The \EasyCrypt pHL statement is written and proved bellow.

\begin{lstlisting}[mathescape,language=easycrypt,xleftmargin=0pt,xrightmargin=0pt,style=easycrypt-default,basicstyle=\scriptsize\ttfamily,morekeywords={}]  
lemma example : phoare [Example.main : true ==> res = 10] <= (1%r / 101%r).
proof.
  proc.
  rnd.
  skip.
  progress.
  have : mu int_distr (fun (x : int) => x * 2 = 10) = 
         mu int_distr (fun (x : int) => x = 5).
    congr.
    rewrite fun_ext /(==).
    move => x.
    smt.
  move => H.
  rewrite H int_distr_mu1. 
  trivial. 
  trivial. 
qed.
\end{lstlisting}

\section{Probabilistic Relational Hoare logic}
\label{sec:prhl}
Probabilistic Relational Hoare logic (pRHL) allows one to write HL
lemmas that compare the execution of two programs. Intuitively, it
allows the proof of statements where two programs are compared and
where users can write pre and postconditions that refer to variables
on both programs. Concretely, variables on the \textit{left} program
can be referred to using the \ec{{1}} tag, whereas variables on the
\textit{right} program can be referred to using the \ec{{2}} tag. A
pRHL triple can be written according to the following syntax.

\begin{lstlisting}[mathescape,language=easycrypt,xleftmargin=0pt,xrightmargin=0pt,style=easycrypt-default,basicstyle=\scriptsize\ttfamily,morekeywords={}]  
lemma p_relational_hoare_logic : equiv [p1 ~ p2 : pre ==> post]. 
\end{lstlisting}

The same tactics that were analyzed in Section~\ref{sec:hoare-logic}
can be applied to pRHL judgments but, instead of consuming statements
in a singe program, it will consume statements on both programs being
analyzed.

\subsection{Relational \ec{rnd} tactic}

The \ec{rnd} tactic, when applied in a pRHL context, it follows a
different behavior when comparing to HL. Concretely

\paragraph{\ec{rnd | rnd f | rnd f g}}

If the conclusion is a pRHL judgement whose programs end with random
assignments \ec{x1 <}$\$$ {d1} and {x2 <}$\$$ {d2}, and {f} and {g}
are functions between the types of {x1} and {x2}, then consume those
random assignments, replacing the conclusion’s postcondition by the
probabilistic weakest precondition of the random assignments wrt. {f}
and {g}. The new postcondition checks that:

\begin{itemize}
\item \ec{f} and \ec{g} are an isomorphism between the distributions
  d1 and d2
\item for all elements \ec{u} in the support of \ec{d1}, the result of
  substituting \ec{u} and \ec{f} u for \ec{x1{1}} and \ec{x2{2}} in
  the conclusion’s original postcondition holds
\end{itemize}

Example: applying \ec{rnd (fun b => if b then 3 else 2) (fun m => m =
  3)} to

\begin{lstlisting}[mathescape,language=easycrypt,xleftmargin=0pt,xrightmargin=0pt,style=easycrypt-default,basicstyle=\scriptsize\ttfamily,morekeywords={}]
n: int
------------------------------------------------------------------------
&1 (left ) : M.h
&2 (right) : N.h

pre = y{2} = n

x <$\$$ {0,1}              (1)  y <- y - 1            
                           (2)  x <$\$$ [2..3]

post = x{1} <=> x{2} + y{2} = n + 2
\end{lstlisting}

\noindent is transformed into

\begin{lstlisting}[mathescape,language=easycrypt,xleftmargin=0pt,xrightmargin=0pt,style=easycrypt-default,basicstyle=\scriptsize\ttfamily,morekeywords={}]
n: int
------------------------------------------------------------------------
&1 (left ) : M.h
&2 (right) : N.h

pre = y{2} = n

x <$\$$ {0,1}              (1)  y <- y - 1            
                           (2)  x <$\$$ [2..3]

post =
  (forall (xR : int),
    xR \in [2..3] => xR = if xR = 3 then 3 else 2) &&
  (forall (xR : int),
    xR \in [2..3] => mu1 [2..3] xR = mu1 {0,1} (xR = 3)) &&
  forall (xL : bool),
    xL \in {0,1} =>
    ((if xL then 3 else 2) \in [2..3]) &&
    xL = ((if xL then 3 else 2) = 3) &&
    (xL <=> (if xL then 3 else 2) + y{2} = n + 2)  
\end{lstlisting}

\subsection{\EasyCrypt pRHL example}
\label{sec:prhl-example}

Consider the following two \EasyCrypt modules

\begin{lstlisting}[mathescape,language=easycrypt,xleftmargin=0pt,xrightmargin=0pt,style=easycrypt-default,basicstyle=\scriptsize\ttfamily,morekeywords={}]  
module type M = {
  proc gen() : int
}.

module Example1 (M : M) = {
  
  proc main(x : int) : int = {
    var r;

    z <@ M.gen()             
    r <$\$$ [0..100]            
    if (x < 100) {           
      y <- x * 2             
    } else {                 
      y <- r + z             
    } 

    return y;
  }
}.

module Example2 (M : M) = {
  
  proc main(x : int) : int = {
    var r;

    z <@ M.gen()             
    r <$\$$ [0..100]            
    if (x < 50) {           
      y <- x * 2             
    } else {                 
      y <- r + z             
    } 

    return y;
  }
}.
\end{lstlisting}

Using pRHL, it is possible to prove that, when the input of both
programs is lower than 50, they will produce the same output.

The \EasyCrypt pRHL statement is written and proved bellow.

\begin{lstlisting}[mathescape,language=easycrypt,xleftmargin=0pt,xrightmargin=0pt,style=easycrypt-default,basicstyle=\scriptsize\ttfamily,morekeywords={}]  
lemma example (M <: M) : equiv [Example1(M).main ~ Example2(M).main : (glob M){1} = (glob M){2} /\ x{1} = x{2} /\ x{1} < 50 ==> res{1} = res{2}]. 
proof.
  proc.
  wp.
  rnd.
  call (_ : true).
  skip.
  progress.
  smt().
  smt().
qed.
\end{lstlisting}

\end{document}

