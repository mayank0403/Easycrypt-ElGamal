\documentclass{article}

\usepackage[english]{babel}
\usepackage[utf8]{inputenc}
\usepackage[T1]{fontenc}

\usepackage{color}
\usepackage{enumerate}
\usepackage{amsmath}
\usepackage{hyperref}
\usepackage{multirow}
\usepackage{float}
\usepackage{graphicx}
\usepackage{mathtools}
\usepackage{amsfonts}
\usepackage[all]{xy}
\usepackage{xspace}
\usepackage{listings,multicol,multirow}
\usepackage{graphicx}
\usepackage{url}
\usepackage{amsmath}
\usepackage{amsfonts}
\usepackage{amsthm}
\usepackage{paralist}
\usepackage{enumitem}
\usepackage{pifont}
\usepackage{chngpage}
\usepackage{fancyhdr}
\usepackage{xcolor}
\usepackage[margin=1in]{geometry}

% Task, TODO and comment env
\newcommand{\comment}[1]{ {\color{blue} $\Rightarrow_{comment}$ #1} }
\newcommand{\taskinl}[1]{ {\color{red} $\bullet$ \textbf{TODO} #1} }
\newcommand{\task}{\item}
\newenvironment{TODO}{%
    \begin{center}
    \color{red}\textbf{Unfinished tasks:}
        \begin{enumerate}
}{%
        \end{enumerate}
    \end{center}
  }

\definecolor{OliveGreen}{rgb}{0,0.6,0}
\definecolor{Blue}{rgb}{0.0, 0.0, 0.6}
\definecolor{Red}{rgb}{0.6, 0.0, 0.0}

\newcommand{\EasyCrypt}{\textsf{EasyCrypt}\xspace}
\newcommand{\OCaml}{\textsf{OCaml}\xspace}
\newcommand{\FSTAR}{\textsf{F$\star$}\xspace}

\newcommand{\INDCPA}{\texttt{IND-CPA}\xspace}

\newtheorem{theorem}{Theorem}[section]
\newtheorem{lemma}[theorem]{Lemma}

% !TEX root = ./main.tex

%\newcommand{\result}{\mathsf{res}}
%\newcommand{\arrayVar}[2][]{{\mathbf{#2}_{#1}}}

\lstdefinelanguage{easycrypt}{
  style=easycrypt-default,
%  procnamekeys={op,pred,fun},
%  procnamestyle={\sffamily\itshape},
  keywordsprefix={'},
  morekeywords=[1]{unit,bool,int,real,list,matrix,option,distr,fset,array,true,false,res, Some, None},
  morekeywords=[2]{Pr,type,op,axiom,lemma,module,pred,const,declare},
  morekeywords=[3]{var,proc,let,in,return},
  morekeywords=[4]{while,if,then,else,fun},
  morekeywords=[5]{theory,end,clone,import,export,as,section,of,proving,with},
  morekeywords=[6]{forall,exists},
  morekeywords=[7]{idtac,change,beta,iota,zeta,logic,delta,simplify,congr,generalize,
                   pose,split,left,right,case,intros,cut,elim,apply,rewrite,elimT,subst,
                   progress,trivial,move, have, wp, rnd, call, skip, rcondt, rcondf},
  morekeywords=[8]{by,assumption,smt,reflexivity, done},
  morekeywords=[9]{first,last,do,try},
  morekeywords=[10]{int_bool_tuple, t, tree},
%  morekeywords=[11]{Rand_t,Distinguisher_t,Evaluator_t,Simulator_t,FRand_t,Adversary_t,MProver_t,MVerifier_t,RandP_t,RandV_t},
%  morekeywords=[12]{Idle,api_input,api_output,api_operator,API,global_state,AStep},
%  morekeywords=[13]{group,t,skey,pkey,plaintext,ciphertext,msg,cipher,Scheme,Adv},
%  moredirectives={prover,print}, % Incomplete
  morecomment=[n][\itshape\color{gray}]{(*}{*)},
  morecomment=[n][\bfseries\color{darkgray}]{(**}{*)}
}

\lstdefinestyle{easycrypt-default}{
  columns=fullflexible,
  captionpos=b,
  frame=tb,
  breaklines=true,
  %xleftmargin=.1\textwidth,
  %xrightmargin=.1\textwidth,
  rangebeginprefix={(**\ begin\ },
  rangeendprefix={(**\ end\ },
  rangesuffix={\ *)},
  includerangemarker=false,
  basicstyle=\small\ttfamily,
  identifierstyle={},
  keywordstyle=[1]{\itshape\color{OliveGreen}},
  keywordstyle=[2]{\bfseries\color{Blue}},
  keywordstyle=[3]{\bfseries\color{Red}},
  keywordstyle=[4]{\bfseries\color{Red}},
  keywordstyle=[5]{\bfseries\color{Blue}},
  keywordstyle=[6]{\itshape\color{Blue}},
  keywordstyle=[7]{\color{Blue}},
  keywordstyle=[8]{\color{Red}},
  keywordstyle=[9]{\color{OliveGreen}},
  keywordstyle=[10]{\itshape\color{OliveGreen}},
  keywordstyle=[11]{\itshape\color{OliveGreen}},
  keywordstyle=[12]{\itshape\color{OliveGreen}},
  keywordstyle=[13]{\itshape\color{OliveGreen}},
  keywordstyle=[14]{\itshape\color{OliveGreen}},
  literate={phi}{{$\!\phi\,$}}1
           {phi1}{{$\!\phi_1$}}1
           {phi2}{{$\!\phi_2$}}1
           {phi3}{{$\!\phi_3$}}1
           {phin}{{$\!\phi_n$}}1
}

\lstdefinestyle{easycrypt-pretty}{
    basicstyle=\small\ttfamily,
    literate={\\in}{{$\mathrel{\in}\;$}}1
              {:=}{{$\mathrel{\gets}$}}1
              {<=}{{$\mathrel{\leq} \; \;$}}1
              {<=}{{$\mathrel{\leq} \; \;$}}1
              {>=}{{$\mathrel{\geq}$}}1
              {<>}{{$\mathrel{\neq} \; \;$}}1
              {=\$}{{$\stackrel{\$}{\gets}\;$}}1
              {forall}{{$\forall\;$}}1
              {exists}{{$\exists\;$}}1
              {->}{{$\rightarrow\;$}}1
              {<r}{$\mathrel{< \! \! \$ \;}$}1
              {<p}{$< \! \! @\;$}1
              {<-}{{$\leftarrow\;$}}1
              {<->}{{$\leftrightarrow\;$}}1
              {<=>}{{$\Leftrightarrow\;$}}1
              {=>}{{$\Rightarrow\;$}}1
              {==>}{{$\Rrightarrow\;$}}1
              {\/\\}{{$\wedge\;$}}1
              {\\\/}{{$\vee\;$}}1
              {.\[}{{[}}1
              {''ora}{{$\mathrel{||}$}}1 %needed for correct display in index
              {'a}{{\color{OliveGreen}$\alpha\,$}}1
              {'b}{{\color{OliveGreen}$\beta\,$}}1
              {'c}{{\color{OliveGreen}$\gamma\,$}}1
              {'t}{{\color{OliveGreen}$\tau\,$}}1
              {'x}{{\color{OliveGreen}$\chi\,$}}1
              {lambda}{{$\lambda\,$}}1
              {sumvd}{{$\sum_{\textrm{v} \in \textrm{d}}$}}1
              {o5}{{$\frac{1}{2}$}}1
              {result}{{$\result$}}1 
              {m0}{{$\textrm{m}_0$}}1
              {m1}{{$\textrm{m}_1$}}1
              {l0}{{$\textrm{l}_0$}}1
}

\newcommand{\ec}[1]{\lstinline[mathescape,language=easycrypt,xleftmargin=0pt,xrightmargin=0pt,style=easycrypt-default,basicstyle=\scriptsize\ttfamily,morekeywords={}]{#1}}


\title{\EasyCrypt\ - Hashed ElGamal semantic security proof exercise}
\date{FMiTF Bootcamp - May 29 - June 2, 2023}
\author{Vitor Pereira}

\begin{document}
\maketitle
%\tableofcontents

\section{Introduction}
\label{sec:introduction}

The purpose of this document is to describe a verification exercise
that consists on performing the security proof of the ElGamal
encryption scheme in its hashed version. We will first give an
overview of how such proofs are done in \EasyCrypt using the
non-hashed ElGamal cryptosystem as an example.

We will closely follow the proof provided at
\cite{cryptoeprint:2004/332}, where the game hopping proof technique
is also explained. The student is encouraged to closely follow
\cite{cryptoeprint:2004/332} while reading this document in order to
match the building blocks described here with the proof described in
\cite{cryptoeprint:2004/332}.

\section{Modeling public key encryption Schemes in \EasyCrypt}
\label{sec:enc-schemes-model}

A public key encryption scheme is a 3-tuple of algorithms:

\begin{itemize}
\item \textsf{KenGen() : pk * sk} - a probabilistic algorithm that
  generates both the public and secret keys (\textsf{pk} and
  \textsf{sk}, respectively).
\item \textsf{Encrypt(pk, pt) : ct} - a probabilistic algorithm that
  takes as input the public key \textsf{pk} and the plaintext
  \textsf{pt} and outputs a ciphertext \textsf{ct}.
\item \textsf{Decrypt(sk, ct) : pt} - a deterministic algorithm that
  takes as input the secret key \textsf{sk} and the ciphertext
  \textsf{ct} and outputs a plaintext \textsf{pt}.  
\end{itemize}

In \EasyCrypt, the modeling of public key encryption schemes can be
captured by a \ec{module type}, as follows

\begin{lstlisting}[mathescape,language=easycrypt,xleftmargin=0pt,xrightmargin=0pt,style=easycrypt-default,basicstyle=\scriptsize\ttfamily,morekeywords={}]  
type pk_t.
type sk_t.
type pt_t.
type ct_t.

module type Scheme = {
  proc key_gen() : pk_t * sk_t
  proc encrypt(pk : pk_t, m : pt_t) : ct_t
  proc decrypt(sk : sk_t, ct : ct_t) : pt_t
}.
\end{lstlisting}

\section{Modeling security definitions in \EasyCrypt}
\label{sec:sec-defs-model}

Cryptographic semantic security, also dubbed as \textit{indistinguishability
against chosen-plaintext attacks} or simply \textit{IND-CPA}, is a
security definition that is captured by the following security
experience (or \textit{game}):

\begin{enumerate}
\item First, an \textit{adversary} (an entity trying to break the
  \textit{IND-CPA} security), chooses two messages, say $m_0$ and $m_1$.
\item Next, the \textit{challenger} selects a random bit $b$ and encrypts
  message $m_0$ if $b = false$ or encrypts $m_1$ if $b = 1$.
\item The \textit{adversary} is then given the encryption of either
  $m_0$ or $m_1$
\item Finally, the \textit{adversary} must output a \textit{decision}
  bit $b'$: it needs to determine if the ciphertext it got from the
  \textit{challenger} corresponds to an encryption of $m_0$ or $m_1$.
\end{enumerate}

If the scheme is \textit{IND-CPA} secure, then the \textit{adversary}
is said to have no \textit{advantage} against the scheme and its best
option is to \textit{guess} which message originated the ciphertext it
received. Formally, a scheme is \textit{IND-CPA} secure if the
probability that the \textit{adversary} outputs $b' = b$ is
$\frac{1}{2}$.

In \EasyCrypt, the \textit{IND-CPA} security game can be defined as a
module as follows

\begin{lstlisting}[mathescape,language=easycrypt,xleftmargin=0pt,xrightmargin=0pt,style=easycrypt-default,basicstyle=\scriptsize\ttfamily,morekeywords={}]  
module type INDCPA_Adv = {
  proc gen_query(pk : pk_t) : pt_t * pt_t
  proc guess(ct : ct_t) : bool
}.

module INDCPA (S : Scheme) (A : INDCPA_Adv) = {
  proc main() : bool = {
    var pk, sk, m0, m1, b, b', ct;

    (pk, sk) <@ S.key_gen();
    (m0, m1) <@ A.gen_query(pk);
    b <$\$$ {0,1};
    ct <@ S.encrypt(pk, if b then m1 else m0);
    b' <@ A.guess(ct);

    return (b = b');
  }
}.
\end{lstlisting}

\section{\EasyCrypt ElGamal specification}
\label{sec:elgamal-spec}

We first provide a brief description of the ElGamal cryptosystem. Let
$G$ be a group of prime order $q$, and let $g \in G$ be a
generator. The key generation algorithm computes $(pk,sk)$ as follows

$$ x \xleftarrow{\$} \mathbb{Z}_q ; pk \gets g^x ; sk \gets x $$

\noindent To encrypt a message $m \in G$, the encryption algorithm
proceeds as follows

$$ y \xleftarrow{\$} \mathbb{Z}_q ; \beta \gets g^y ; \delta \gets pk^y ; \zeta \gets
\delta \cdot m ; ct \gets (\delta, \zeta)$$

Finally, decryption of a ciphertext is simply done by

$$ m <- \zeta / \beta^{sk} $$

% \subsection{Finite field \EasyCrypt library}



% \subsection{Cyclic group \EasyCrypt library}

% \subsection{\EasyCrypt ElGamal definition}

The ElGamal encryption scheme can be defined in \EasyCrypt as a module
of type \ec{Scheme} in accordance to the following code.

\begin{lstlisting}[mathescape,language=easycrypt,xleftmargin=0pt,xrightmargin=0pt,style=easycrypt-default,basicstyle=\scriptsize\ttfamily,morekeywords={}]  
module ElGamal : Scheme = {
  proc key_gen() : pk_t * sk_t = {
    var pk, sk;

    sk <$\$$ FDistr.dt;
    pk <- g ^ sk;

    return (pk, sk);
  }

  proc encrypt(pk : pk_t, m : pt_t) : ct_t = {
    var y, bet, delt, zet;

    y <$\$$ FDistr.dt;
    bet <- g ^ y;
    delt <- pk ^ y;
    zet <- delt * m;

    return (bet, zet);
  }

  proc decrypt(sk : sk_t, ct : ct_t) : pt_t = {
    var m, bet, zet;

    (bet, zet) <- ct;
    m <- zet / (bet ^ sk);

    return (m);
  }
}.
\end{lstlisting}

\noindent where \ec{FDistr.dt} is a uniform probability distribution
over elements of the finite field $\mathbb{Z}_q$.

Finite field and cyclic group arithmetic libraries are provided in the
\EasyCrypt scripts that accompany this document.

\section{ElGamal semantic security proof}
\label{sec:elgamal-proof}

ElGamal encryption is semantically secure under the Decisional
Diffie-Hellman (DDH) assumption. This is the assumption that it is
hard to distinguish triples of the form $(g^x, g^y, g^{x*y})$ from triples of
the form $(g^x, g^y, g^{z})$, where $x$, $y$, and $z$ are random
elements of $\mathbb{Z}_q$.

\subsection{DDH assumption in \EasyCrypt}

The DDH assumption is formulated in \EasyCrypt as follows

\begin{lstlisting}[mathescape,language=easycrypt,xleftmargin=0pt,xrightmargin=0pt,style=easycrypt-default,basicstyle=\scriptsize\ttfamily,morekeywords={}]  
module type Adversary = {
  proc guess(gx gy gz : group) : bool
}.

module DDH0 (A : Adversary) = {
  proc main() : bool = {
    var b, x, y;

    x <$\$$ FDistr.dt;
    y <$\$$ FDistr.dt;
    b <@ A.guess(g ^ x, g ^ y, g ^ (x * y));
    return b;
  }
}.

module DDH1 (A : Adversary) = {
  proc main() : bool = {
    var b, x, y, z;

    x <$\$$ FDistr.dt;
    y <$\$$ FDistr.dt;
    z <$\$$ FDistr.dt;
    b <@ A.guess(g ^ x, g ^ y, g ^ z);
    return b;
  }
}.
\end{lstlisting}

\noindent where the \textit{advantage} against the DDH is formulated
as

$$ | Pr [DDH0(A).main : b = 1 ] - Pr [DDH1(A).main : b = 1 ] | $$

\subsection{DDH reduction proof}

Cryptographic reduction proof are done following a
\textit{contradiction} approach. For example, to reduce the security
of the ElGamal encryption scheme to the DDH assumption, one needs to
prove that any procedure that can be used to break the security of the
scheme can also be used to break the DDH assumption. However, DDH is
considered a hard mathematical problem and, therefore, no algorithm
can be used to solve it in computational time, meaning that we arrived 
at a contradiction. Consequently, there is no \textit{adversary} is
able to successfully break the security of ElGamal, since that would
mean that the DDH assumption would not be hard problem, as it is
assumed.

To perform the reduction proof in \EasyCrypt, one needs to write a DDH
\textit{adversary} that uses the \textit{IND-CPA} adversary as a
sub-routine. That adversary is written bellow.

\begin{lstlisting}[mathescape,language=easycrypt,xleftmargin=0pt,xrightmargin=0pt,style=easycrypt-default,basicstyle=\scriptsize\ttfamily,morekeywords={}]  
module D (A : INDCPA_Adv) = {
  proc guess(gx gy gz : group) : bool = {
    var m0, m1, b, b';

    (m0, m1) <@ A.gen_query(gx);
    b <$\$$ {0,1};
    b' <@ A.guess(gy, gz * (if b then m1 else m0));

    return (b = b');
  }
}.
\end{lstlisting}

\noindent The \ec{D} \textit{adversary} first queries \ec{A} (the
\textit{IND-CPA} \textit{adversary}) to generate two messages and then
uses \ec{A} \ec{guess} procedure (the procedure used by \ec{A} to
break ElGamal's security) to procedure its own decision bit.

Next, we define a new \textit{game} that is equivalent to the original
\textit{IND-CPA} experience, except that the value $\zeta$ is
calculated as $\zeta \gets g^z$, with random $z \in \mathbb{Z}_q$,
instead of $\zeta \gets \delta \cdot m$.

\begin{lstlisting}[mathescape,language=easycrypt,xleftmargin=0pt,xrightmargin=0pt,style=easycrypt-default,basicstyle=\scriptsize\ttfamily,morekeywords={}]  
module Game1 = {
  proc main() : bool = {
    var pk, sk, m0, m1, b, b', ct, y, bet, delt, z, zet;

    (pk, sk) <@ ElGamal.key_gen();
    (m0, m1) <@ A.gen_query(pk);
    b <$\$$ {0,1};

    y <$\$$ FDistr.dt;
    bet <- g ^ y;
    delt <- pk ^ y;
    z <$\$$ FDistr.dt;
    zet <- g ^ z;
    ct <- (bet, zet);

    b' <@ A.guess(ct);

    return (b = b');
  }
}.
\end{lstlisting}

Now, we are able to prove that the \ec{D} \textit{adversary}
interpolates both the \textit{IND-CPA} \textit{game} (which is
typically called \textit{Game0}) and \ec{Game1}. That is done by
proving the following two equivalence lemmas

\begin{lstlisting}[mathescape,language=easycrypt,xleftmargin=0pt,xrightmargin=0pt,style=easycrypt-default,basicstyle=\scriptsize\ttfamily,morekeywords={}]  
lemma game0_ddh0_equiv : equiv [ DDH0(D(A)).main ~ Game0(A).main : ={glob A} ==> ={res} ].

lemma game1_ddh1_equiv : equiv [ DDH1(D(A)).main ~ Game1.main : ={glob A} ==> ={res} ].
\end{lstlisting}

\noindent and, since both games are equivalent, they output the same
bit with equivalent probability

\begin{lstlisting}[mathescape,language=easycrypt,xleftmargin=0pt,xrightmargin=0pt,style=easycrypt-default,basicstyle=\scriptsize\ttfamily,morekeywords={}]  
lemma game0_ddh0_pr &m : Pr [ DDH0(D(A)).main() @ &m : res ] = Pr [ Game0(A).main() @ &m : res ].
      
lemma game1_ddh1_pr &m : Pr [ DDH1(D(A)).main() @ &m : res ] = Pr [ Game1.main() @ &m : res ].
\end{lstlisting}

The complete proof of these lemmas can be found in the accompanying
\EasyCrypt files.

Note that the output of \ec{Game1} (\ec{b = b'}) is
independent of the bit \ec{b}, since no message is actually
encrypted. Therefore, one can prove that the probability of the event
\ec{b = b'} is precisely $\frac{1}{2}$.

\begin{lstlisting}[mathescape,language=easycrypt,xleftmargin=0pt,xrightmargin=0pt,style=easycrypt-default,basicstyle=\scriptsize\ttfamily,morekeywords={}]  
lemma game1_pr &m : Pr [ Game1.main() @ &m : res ] = 1%r / 2%r.
\end{lstlisting}

Finally, we can establish the concrete security bounds of the ElGamal
encryption scheme by proving the following lemma

\begin{lstlisting}[mathescape,language=easycrypt,xleftmargin=0pt,xrightmargin=0pt,style=easycrypt-default,basicstyle=\scriptsize\ttfamily,morekeywords={}]  
lemma security &m :
  Pr [ INDCPA(ElGamal, A).main() @ &m : res ] = 
    1%r/2%r + (Pr[ DDH0(D(A)).main() @ &m : res ] - Pr[ DDH1(D(A)).main() @ &m : res ]).
\end{lstlisting}

\noindent where \ec{Pr[ DDH0(D(A)).main() @ &m : res ] - Pr[
  DDH1(D(A)).main() @ &m : res ]} represents the (negligible)
advantage against the DDH assumption.

\section{Exercise: semantic security proof of hashed ElGamal}
\label{sec:exercise}

Your goal is to take advantage of the description of the semantic
security proof of ElGamal described above and detailed in the
\EasyCrypt files that are attached to this document, and perform the
semantic security of the ElGamal encryption scheme in its hashed
version.

Briefly, the security of hashed ElGamal can be reduced to the DDH
assumption and to the \textit{entropy smoothing} assumption of the
underlying hash function. The first reduction step is the same of the
non-hashed ElGamal version. You will need to do the second reduction
proof, i.e., the reduction to the \textit{entropy smoothing} hash
function assumption. You will not need to formalize this assumption,
as it is part of the \EasyCrypt set of files available to
you. Similarly to the DDH reduction proof, you will need to write an
\textit{entropy smoothing} adversary based on an \textit{IND-CPA}
adversary and write a new game hop that interpolates with the this new
assumption.

The complete security proof of the hashed ElGamal cryptosystem can
also be found at \cite{cryptoeprint:2004/332}.

\bibliographystyle{plain}
\bibliography{refs.bib}

\end{document}